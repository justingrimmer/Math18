\documentclass{beamer}

%\usepackage[table]{xcolor}
\mode<presentation> {
  \usetheme{Boadilla}
%  \usetheme{Pittsburgh}
%\usefonttheme[2]{sans}
\renewcommand{\familydefault}{cmss}
%\usepackage{lmodern}
%\usepackage[T1]{fontenc}
%\usepackage{palatino}
%\usepackage{cmbright}
  \setbeamercovered{transparent}
\useinnertheme{rectangles}
}
%\usepackage{normalem}{ulem}
%\usepackage{colortbl, textcomp}
\setbeamercolor{normal text}{fg=black}
\setbeamercolor{structure}{fg= black}
\definecolor{trial}{cmyk}{1,0,0, 0}
\definecolor{trial2}{cmyk}{0.00,0,1, 0}
\definecolor{darkgreen}{rgb}{0,.4, 0.1}
\usepackage{array}
\beamertemplatesolidbackgroundcolor{white}  \setbeamercolor{alerted
text}{fg=red}

\setbeamertemplate{caption}[numbered]\newcounter{mylastframe}

%\usepackage{color}
\usepackage{tikz}
\usetikzlibrary{arrows}
\usepackage{colortbl}
%\usepackage[usenames, dvipsnames]{color}
%\setbeamertemplate{caption}[numbered]\newcounter{mylastframe}c
%\newcolumntype{Y}{\columncolor[cmyk]{0, 0, 1, 0}\raggedright}
%\newcolumntype{C}{\columncolor[cmyk]{1, 0, 0, 0}\raggedright}
%\newcolumntype{G}{\columncolor[rgb]{0, 1, 0}\raggedright}
%\newcolumntype{R}{\columncolor[rgb]{1, 0, 0}\raggedright}

%\begin{beamerboxesrounded}[upper=uppercol,lower=lowercol,shadow=true]{Block}
%$A = B$.
%\end{beamerboxesrounded}}
\renewcommand{\familydefault}{cmss}
%\usepackage[all]{xy}

\usepackage{tikz}
\usepackage{lipsum}

 \newenvironment{changemargin}[3]{%
 \begin{list}{}{%
 \setlength{\topsep}{0pt}%
 \setlength{\leftmargin}{#1}%
 \setlength{\rightmargin}{#2}%
 \setlength{\topmargin}{#3}%
 \setlength{\listparindent}{\parindent}%
 \setlength{\itemindent}{\parindent}%
 \setlength{\parsep}{\parskip}%
 }%
\item[]}{\end{list}}
\usetikzlibrary{arrows}
%\usepackage{palatino}
%\usepackage{eulervm}
\usecolortheme{lily}

\newtheorem{com}{Comment}
\newtheorem{lem} {Lemma}
\newtheorem{prop}{Proposition}
\newtheorem{thm}{Theorem}
\newtheorem{defn}{Definition}
\newtheorem{cor}{Corollary}
\newtheorem{obs}{Observation}
 \numberwithin{equation}{section}


\title[Methodology I] % (optional, nur bei langen Titeln nötig)
{Math Camp}

\author{Justin Grimmer}
\institute[Stanford University]{Professor\\Department of Political Science \\  Stanford University}
\vspace{0.3in}

\date{September 19th, 2017}

\begin{document}

\begin{frame}
\maketitle
\end{frame}

\begin{frame}


Questions?\\


\end{frame}




\begin{frame}


Today:
\begin{itemize}
\item[1)] Properties of Expectations
\item[2)] Changing Coordinates
\item[3)] Moment Generating Functions
\item[4)] Inequalities 
\item[5)] Convergence
\end{itemize}



\end{frame}




\begin{frame}
\frametitle{Iterated Expectations}

\begin{prop}
Suppose $X$ and $Y$ are random variables.  Then 

\begin{eqnarray}
E[X] & = & E[E[X|Y]] \nonumber 
\end{eqnarray}




\end{prop}


\begin{itemize}
\item[-] Inner Expectation is $E[X|Y] = \int_{-\infty}^{\infty} x f_{X|Y} (x|y) dx$.  
\item[-] Outer expectation is over $y$.  
\end{itemize}


\end{frame}


\begin{frame}
\frametitle{Iterated Expectations}

\pause 
\begin{proof}
\begin{eqnarray}
\invisible<1>{E[E[X|Y]] & = & \int_{-\infty}^{\infty}  \int_{-\infty}^{\infty} x f_{X|Y} (x|y) f_{Y}(y) dx dy \nonumber \\} \pause 
 \invisible<1-2>{&= & \int_{-\infty}^{\infty}  \int_{-\infty}^{\infty} x f_{X|Y} (x|y) f_{Y}(y) dy dx \nonumber \\} \pause 
 \invisible<1-3>{& = & \int_{-\infty}^{\infty} x \int_{-\infty}^{\infty}  f(x, y) dy dx  \nonumber \\} \pause 
 \invisible<1-4>{& = & \int_{-\infty}^{\infty}  x f_{X}(x) dx  \nonumber \\} \pause 
  \invisible<1-5>{& =& E[X] \nonumber } 
\end{eqnarray}



\end{proof}



\end{frame}


\begin{frame}
\frametitle{Iterated Expectations}

\begin{defn}
Suppose $Y$ is a continuous random variable with $Y \in [0,1]$ and pdf of $Y$ given by 

\begin{eqnarray}
f(y) & = & \frac{\Gamma(\alpha_1 + \alpha_2)}{\Gamma(\alpha_{1} ) \Gamma(\alpha_{2})} y^{\alpha_{1} - 1} (1- y)^{\alpha_{2} - 1 } \nonumber 
\end{eqnarray}

Then we will say $Y$ is a \alert{Beta} distribution with parameters $\alpha_{1}$ and $\alpha_{2}$.  Equivalently,
\begin{eqnarray}
Y & \sim & \text{Beta}(\alpha_{1}, \alpha_{2} ) \nonumber 
\end{eqnarray}



\end{defn}


\begin{itemize}
\item[-] Beta is a distribution on \alert{proportions} 
\item[-] Beta is a special case of the \alert{Dirichlet} distribution 
\item[-] $E[Y] = \frac{\alpha_{1}}{\alpha_{1} + \alpha_{2}}$
\end{itemize}


\end{frame}


\begin{frame}
\frametitle{Iterated Expectations}

Suppose 
\begin{eqnarray}
\pi & \sim & \text{Beta}(\alpha_{1}, \alpha_{2}) \nonumber \\
Y|\pi, n & \sim & \text{Binomial}(n, \pi)\nonumber 
\end{eqnarray}

What is $E[Y]$? \pause 

\begin{eqnarray}
\invisible<1>{E[Y] & = & E[E[Y| \pi]] \nonumber \\} \pause 
\invisible<1-2>{& = & \int_{-\infty}^{\infty} \sum_{j = 0}^{N} {{N}\choose{j}} j p(j|\pi) f(\pi) d\pi \nonumber \\} \pause 
\invisible<1-3>{ & = & \int_{-\infty}^{\infty} N \pi f(\pi) d\pi \nonumber  \\} \pause 
  \invisible<1-4>{& = & N \frac{\alpha_{1}}{\alpha_{1} + \alpha_{2}} \nonumber } 
\end{eqnarray}


\end{frame}



\begin{frame}
\frametitle{Change of Coordinates}


\begin{prop}
Suppose $X$ is a random variable and $Y = g(X)$, where $g:\Re \rightarrow \Re$ that is a $monotonic$ function.  \\
Define $g^{-1}:\Re \rightarrow \Re$ such that $g^{-1}(g(X)) = X$ and is differentiable.  Then, 
\begin{eqnarray}
f_{Y}(y) & = & f_{X}(g^{-1}(y))\left|\frac{\partial g^{-1}(y)}{\partial y} \right| \text{ if $y = g(x)$ for some $x$ } \nonumber \\
& = & 0 \text{ otherwise } \nonumber 
\end{eqnarray}

\end{prop}


\end{frame}

\begin{frame}
\frametitle{Change of Coordinates}


\begin{proof}

Suppose $g(\cdot)$ is monotonically increasing (WLOG)\pause 
\begin{eqnarray}
\invisible<1>{F_{Y}(y) & = & P(Y \leq y) \nonumber \\} \pause 
\invisible<1-2>{& = & P(g(X) \leq y) \nonumber \\} \pause 
\invisible<1-3>{& = & P(X \leq g^{-1}(y) ) \nonumber \\} \pause 
\invisible<1-4>{& = & F_{X}(g^{-1}(y) ) \nonumber } \pause 
\end{eqnarray}
\invisible<1-5>{Now differentiating to get the pdf } \pause 
\begin{eqnarray}
\invisible<1-6>{\frac{\partial F_{Y}(y)}{\partial y} & = & \frac{\partial F_{X}(g^{-1}(y) )} {\partial y } \nonumber \\
 & = & f_{X}(g^{-1}(y) ) \frac{\partial g^{-1}(y)}{\partial y }   \nonumber } \pause 
\end{eqnarray}
\invisible<1-7>{Then this is a pdf because $\frac{\partial g^{-1}(Y)}{\partial y } > 0$.  } 


\end{proof}



\end{frame}


\begin{frame}
\frametitle{Change of Coordinates}

Suppose $X$ is a random variable with pdf $f_{X}(x)$.  Suppose $Y =  X^{n}$.  Find $f_{Y}(y)$. \pause  \\
\invisible<1>{Then $g^{-1} (x)  =  x^{1/n}$.  } \pause 




\begin{eqnarray}
\invisible<1-2>{f_{Y}(y) & = & f_{X}(g^{-1}(y)) \left| \frac{\partial g^{-1}(Y)}{\partial y } \right| \nonumber \\} \pause 
\invisible<1-3>{ & = & f_{X} (y^{1/n} ) \frac{y^{\frac{1}{n} - 1}}{n} \nonumber } 
\end{eqnarray}
\pause
\invisible<1-4>{\alert{We've used this to derive many of the pdfs}} \pause 
\begin{itemize}
\invisible<1-5>{\item[-] Normal distribution }
\invisible<1-5>{\item[-] Chi-Squared Distribution}
\end{itemize}




\end{frame}







\begin{frame}
\frametitle{Moment Generating Functions}


\begin{defn}
Suppose $X$ is a random variable with pdf $f$.  Define, 
\begin{eqnarray}
E[X^{n}] & = & \int_{-\infty}^{\infty} x^{n} f(x) dx   \nonumber 
\end{eqnarray}

We will call $X^{n}$ the \alert{$n^{\text{th}}$} moment of $X$ 

\end{defn}

\begin{itemize}
\item[-] By this definition $var(X) = \text{Second Moment} - \text{First Moment}^{2} $
\item[-] We are assuming that the integral converges
\end{itemize}

\end{frame}


\begin{frame}
\frametitle{Moment Generating Functions}

\begin{prop}
Suppose $X$ is a random variable with pdf $f(x)$.  Call $M(t) = E[e^{tX}]$, 
\begin{eqnarray}
M(t) & = & E[e^{tX}] \nonumber \\
 & = & \int_{-\infty}^{\infty} e^{tx} f(x) dx \nonumber 
 \end{eqnarray}

We will call $M(t)$ the moment generating function, because:
\begin{eqnarray}
\frac{\partial^{n} M (t) }{\partial^{n} t}|_{0} & = & E[X^{n}] \nonumber 
\end{eqnarray}

(Assuming that we can interchange derivative and integral)

\end{prop}



\end{frame}




\begin{frame}

\scalebox{0.25}{\includegraphics{Meme.jpg}}

\end{frame}



\begin{frame}
\frametitle{Moment Generating Functions}

\begin{small}
\begin{proof}
Recall the Taylor Expansion of $e^{tX}$ at $0$, \pause 
\begin{eqnarray}
\invisible<1>{e^{tX} & = & 1 + tx + \frac{t^2 x^2}{2!} + \frac{t^3 x^3}{3!} + \hdots \nonumber } \pause 
\end{eqnarray}

\invisible<1-2>{Then, } \pause 
\begin{eqnarray}
\invisible<1-3>{E[e^{tX} ] & = &  1 + tE[X] + \frac{t^2}{2!} E[X^2] + \frac{t^3}{3!} E[X^3] + \hdots \nonumber } \pause 
\end{eqnarray}

\invisible<1-4>{Differentiate once:} \pause 
\begin{eqnarray}
\invisible<1-5>{\frac{\partial M(t)}{\partial t} & = & 0 + E[X]  + \frac{2t}{2!} E[X^2] + \hdots \nonumber \\
M^{'}(0) & = & 0 + E[X] + 0 + 0 \hdots \nonumber } 
\end{eqnarray}





\end{proof}

\end{small}



\end{frame}

\begin{frame}

\begin{small}
\begin{proof}

Differentiate $n$ times \pause 
\begin{eqnarray}
\invisible<1>{\frac{\partial^{n} M(t)}{\partial^{n} t } & = & 0 + 0 + 0 + \hdots + \frac{n \times n-1 \times \hdots 2 \times t^{0} E[X^{n}] }{n!}  + \frac{n!t E[X^{n+1}] }{(n+ 1)! } + \hdots \nonumber \\} \pause 
\invisible<1-2>{& = & \frac{n! E[X^{n}] }{n!}  + \frac{n!t E[X^{n+1}] }{(n+ 1)! } + \hdots  \nonumber } \pause 
\end{eqnarray}

\invisible<1-3>{Evaluated at 0, yields $M^{n}(0)  = E[X^{n}] $} \pause 

\end{proof}

\begin{itemize}
\invisible<1-4>{\item[-] If two random variables, $X$ and $Y$ have the same moment generating functions, then $F_{X}(x) = F_{Y}(x)$ for \alert{almost all} $x$.  } 
\end{itemize}


\end{small}
\end{frame}


\begin{frame}
\frametitle{The Moments of the Normal Distribution}
Suppose $Z \sim N(0,1)$.  \pause 

\begin{eqnarray}
\invisible<1>{E[e^{tX}] & = & \frac{1}{\sqrt{2\pi}} \int_{-\infty}^{\infty} e^{tx} e^{-x^2/2} dx \nonumber } \pause 
\end{eqnarray}

\invisible<1-2>{$tx - \frac{1}{2} x^2 = -\frac{1}{2}\left( ( x - t)^2 - t^2 \right)$} \pause 

\begin{eqnarray} 
\invisible<1-3>{E[e^{tX}] & = & \frac{1}{\sqrt{2\pi}}  e^{\frac{t^{2}}{2}}\int_{-\infty}^{\infty} e^{-(x- t)^2/2} dx \nonumber \\} \pause 
 \invisible<1-4>{& = & e^{\frac{t^{2}}{2}} \nonumber } 
\end{eqnarray}


\end{frame}

\begin{frame}
\frametitle{Extracting Moments of the Normal Distribution}

\pause 
\begin{eqnarray}
\invisible<1>{M^{'}(0) & = &E[X] =  e^{t^2/2} t |_{0} = 0 \nonumber \\} \pause
\invisible<1-2>{M^{''} (0 ) & = & E[X^2] =  e^{t^2/2} (t^2 + 1) |_{0} = 1 \nonumber \\} \pause 
\invisible<1-3>{M^{'''} (0) & = & E[X^3] = e^{t^2/2} t (t^2 + 3) |_{0} = 0 \nonumber \\} \pause 
\invisible<1-4>{M^{''''} (0) & = & E[X^4] = e^{t^2/2} (t^4 + 6t^2 + 3)|_{0} = 3 \nonumber } \\ \pause 
\invisible<1-5>{M^{5} (0) & = & E[X^5] = e^{t^2/2} t (t^4 + 10t^2 + 15)|_{0} = 0 \nonumber } \\ \pause 
\invisible<1-6>{M^{6} (0) & = & E[X^6] = e^{t^2/2} (t^6 + 15t^4 + 45t^2 + 15 )|_{0} = 15 \nonumber } \pause 
\end{eqnarray} 


\end{frame}

\begin{frame}

\begin{prop}
Suppose $X_{i}$ are a sequence of independent random variables.  Define 
\begin{eqnarray}
Y & = & \sum_{i=1}^{N} X_{i} \nonumber 
\end{eqnarray}

Then 
\begin{eqnarray}
M_{Y}(t) & = & \prod_{i=1}^{N} M_{X_{i}}(t) \nonumber 
\end{eqnarray}


\end{prop}

\end{frame}

\begin{frame}

\begin{proof}
\begin{eqnarray}
M_{Y}(t) & = & E[e^{tY}] \nonumber \\ \pause 
\invisible<1>{& = & E[e^{t\sum_{i=1}^{N} X_{i} } ] \nonumber \\ } \pause 
 \invisible<1-2>{& = & E[e^{tX_{1} + tX_{2} + \hdots tX_{N} } ] \nonumber \\} \pause 
  \invisible<1-3>{& = & E[e^{tX_{1} }]E[e^{tX_{2} }] \hdots E[e^{tX_{N} }] \text{ (by independence) } \nonumber \\} \pause 
   \invisible<1-4>{& = & \prod_{i=1}^{N} E[e^{tX_{i}}] \nonumber } 
\end{eqnarray}

\end{proof}

\end{frame}






\begin{frame}
\frametitle{Inequalities and Limit Theorems}


\alert{Limit Theorems}
\begin{itemize}
\item[-] What happens when we consider a long sequence of random variables ?
\item[-] What can we reasonably infer from data?
\begin{itemize}
\item[-] Laws of large numbers: averages of random variables converge on expected value?
\item[-] Central Limit Theorems: sum of random variables have normal distribution?
\end{itemize}
\item[-] We'll focus on intuition for both, but we'll prove some stuff too.  
\end{itemize}



\end{frame}


\begin{frame}
\frametitle{Weak Law of Large Numbers}

Proof plan: 
\begin{itemize}
\item[-] Markov's Inequality
\item[-] Chebyshev's Inequality
\item[-] Weak Law of Large Numbers 
\end{itemize}


\end{frame}



\begin{frame}
\frametitle{Markov's Inequality}

\begin{prop}
Suppose $X$ is a random variable that takes on non-negative values.  Then, for all $a>0$, 
\begin{eqnarray}
P(X\geq a) & \leq & \frac{E[X]}{a} \nonumber 
\end{eqnarray}
\end{prop}

\end{frame}


\begin{frame}
\frametitle{Markov's Inequality} 


\begin{proof}
For $a>0$,  \pause 
\begin{eqnarray}
\invisible<1>{E[X] & = & \int_{0}^{\infty} x f(x) dx \nonumber \\} \pause 
 \invisible<1-2>{& = & \int_{0}^{a} x f(x) dx + \int_{a}^{\infty} x f(x) dx \nonumber } \pause 
\end{eqnarray}

\invisible<1-3>{Because $X\geq 0$, } \pause 

\begin{eqnarray}
\invisible<1-4>{E[X] \geq \int_{a}^{\infty} x f(x) dx  \geq \int_{a}^{\infty} a f(x)dx = a P(X \geq a )\nonumber \\} \pause 
\invisible<1-5>{\frac{E[X]}{a} \geq P(X \geq a )\nonumber } 
\end{eqnarray}


\end{proof}



\end{frame}

\begin{frame}
\frametitle{Chebyshev's Inequality} 

\begin{prop} 
If $X$ is a random variable with mean $\mu$ and variance $\sigma^2$, then, for any value $k>0$, 
\begin{eqnarray}
P(|X - \mu| \geq k) & \leq & \frac{\sigma^2}{k^2} \nonumber 
\end{eqnarray}

\end{prop}


\end{frame}

\begin{frame}
\frametitle{Chebyshev's Inequality} 


\begin{proof}
Define the random variable 
\begin{eqnarray}
 Y & = & (X - \mu)^2 \nonumber 
\end{eqnarray}

Where $\mu = E[X]$. 

\pause 

\invisible<1>{Then we know $Y$ is a non-negative random variable.  Set $a = k^2$.  \\} \pause 
\invisible<1-2>{Applying the inequality:} \pause 
\begin{eqnarray}
\invisible<1-3>{P(Y \geq k^2) \leq \frac{E[Y]}{k^2} \nonumber \\} \pause 
\invisible<1-4>{P( (X- \mu)^2 \geq k^2) \leq \frac{E[(X - \mu)^2]}{k^2} \nonumber \\} \pause 
\invisible<1-5>{P( (X- \mu)^2 \geq k^2) \leq \frac{\sigma^2}{k^2}\nonumber } 
\end{eqnarray}

\end{proof}

\end{frame}


\begin{frame}
\frametitle{Chebyshev's Inequality} 


Further we know that, 
\begin{eqnarray}
(X - \mu)^2  \geq k^2 \nonumber 
\end{eqnarray}

\pause 

\invisible<1>{Implies that } \pause 
\begin{eqnarray}
\invisible<1-2>{|X - \mu| \geq k \nonumber } \pause 
\end{eqnarray}

\invisible<1-3>{Thus, we have shown } \pause 
\begin{eqnarray}
\invisible<1-4>{P( |X- \mu| \geq k) \leq \frac{\sigma^2}{k^2}\nonumber } 
\end{eqnarray}

\end{frame}



\begin{frame}
\frametitle{Sequence of Random Variables}

Sequence of Independent and Identically, Distributed Random variables.
\begin{itemize}
\item[-] Sequence: $X_{1}, X_{2}, \hdots, X_{n}, \hdots $
\item[-] Think of a sequence as sampled \alert{data}:
\begin{itemize}
\item[-] Suppose we are drawing a sample of $N$ observations
\item[-] Each observation will be a random variable, say $X_{i}$
\item[-] With realization $x_{i}$ 
\end{itemize}
\end{itemize}

\end{frame}


\begin{frame}
\frametitle{Mean/Variance of Sample Mean}

\begin{prop}
Let $X_{1}, X_{2}, \hdots, X_{n}$ be a random sample from a distribution with mean $\mu$ and variance $\sigma^2$.  Let $\bar{X}_{n}$ be the sample mean.  Then 
$E[\bar{X}_{n}] = \mu$ and var$(\bar{X}_{n}) = \frac{\sigma^2}{n}$
\end{prop}

\pause 
\begin{proof}

\begin{eqnarray}
\invisible<1>{E[\bar{X}_{n}] & = & \frac{1}{n}\sum_{i=1}^{n} E[X_{i}] \nonumber \\} \pause 
\invisible<1-2>{& = & \frac{1}{n} n \mu  = \mu \nonumber } 
\end{eqnarray}



\end{proof}


\end{frame}


\begin{frame}
\frametitle{Mean/Variance of Sample Mean}


\pause 
\begin{eqnarray}
\invisible<1>{\text{var}(\bar{X}_{n}) & = & \frac{1}{n^2} \text{var}(\sum_{i=1}^{n} X_{i}) \nonumber \\} \pause 
 \invisible<1-2>{&= & \frac{1}{n^2} \sum_{i=1}^{n} \text{var}(X_{i}) \nonumber \\} \pause 
  \invisible<1-3>{& = & \frac{1}{n^2} n \sigma^2 = \frac{\sigma^2}{n} \nonumber } 
\end{eqnarray}


\end{frame}



\begin{frame}
\frametitle{Weak Law of Large Numbers} 

\begin{prop}
Suppose $X_{1}, X_{2}, \hdots, X_{n}$ is a random sample from a distribution with mean $\mu$ and $Var(X_{i})= \sigma^2$.  Then, for all $\epsilon >0$, 
\begin{eqnarray}
P\left\{ \left| \frac{X_{1} + X_{2} + \hdots + X_{n} }{n} -\mu \right| \geq \epsilon \right\} \rightarrow 0 \text{ as } n \rightarrow \infty \nonumber 
\end{eqnarray}

\end{prop}


\end{frame}


\begin{frame}
\frametitle{Weak Law of Large Numbers}

\begin{proof}
From our previous proposition \pause 
\begin{eqnarray}
\invisible<1>{\frac{\text{E}[X_{1} + X_{2} + \cdots + X_{n} ]}{n} & = & \frac{\sum_{i=1}^{n} E[X_{i}] }{n}  = \mu \nonumber} \pause 
\end{eqnarray}
\invisible<1-2>{Further, } \pause 
\begin{eqnarray}
\invisible<1-3>{\text{E}[ (\frac{\sum_{i=1}^{n} X_{i} - \mu}{n} )^2] = \frac{\text{Var}(X_{1} + X_{2} + \cdots + X_{n} )}{n^2} & = & \frac{ \sum_{i=1}^{n} \text{Var}(X_{i}) }{n^2} = \frac{\sigma^2}{n} \nonumber } \pause 
\end{eqnarray}
\invisible<1-4>{Apply Chebyshev's Inequality: } \pause 
\begin{eqnarray}
\invisible<1-5>{P\left\{ \left| \frac{X_{1} + X_{2} + \hdots + X_{n} }{n} -\mu \right| \geq \epsilon \right\}\leq \frac{\sigma^2}{n \epsilon^2} \nonumber }
\end{eqnarray}

\end{proof}


\end{frame}

\begin{frame}
Suppose $X_{1}, X_{2}, \hdots$ are iid normal distributions,
\begin{eqnarray}
X_{i} & \sim & \text{Normal}(0, 10) \nonumber 
\end{eqnarray}

\begin{eqnarray} 
P\left\{ \left| \frac{X_{1} + X_{2} + \hdots + X_{n} }{n} -\mu \right| \geq 0.1 \right\} \text{ as } n \rightarrow \infty\nonumber 
\end{eqnarray}
\pause 


\only<1-2>{\invisible<1>{\begin{center}
\scalebox{0.4}{\includegraphics{c8fig.pdf}}
\end{center}

}

}

\only<3-6>{Suppose we want to guarantee that we have at most a $0.01$ probability of being more than $0.1$ away from the true $\mu$.  How big do we need $n$?\pause 
\begin{eqnarray}
\invisible<1-3>{0.01 & = & \frac{10}{n (0.1^2) } \nonumber} \pause  \\
\invisible<1-4>{n & = & \frac{1000}{0.01} \nonumber } \pause \\
\invisible<1-5>{n & = & 100,000 } \nonumber 
\end{eqnarray}
}


\end{frame}

%
%\begin{frame}
%\frametitle{Strong Law of Large Numbers}
%Statement\\
%Difference: this says equality, other says that it gets really close (but fails to establish equality) \\
%Give intuition for this 
%\end{frame}
%
%
%
%\begin{frame}
%\frametitle{Central Limit Theorem} 
%
%Statement, intuition, and examples.  
%
%
%\end{frame}



\begin{frame}
\frametitle{Sequences and Convergence} 

Sequence (refresher):\\
\begin{eqnarray}
\left\{a_{i} \right\}_{i=1}^{\infty} & =  & \left\{a_{1}, a_{2}, a_{3}, \hdots, a_{n}, \hdots, \right\} \nonumber 
\end{eqnarray}


\begin{defn}
We say that the sequence $\left\{a_{i} \right\}_{i=1}^{\infty} $ converges to real number $A$ if for each $\epsilon>0$ there is a positive integer $N$ such that for $n\geq N$, $|a_{n} - A| < \epsilon$
\end{defn}


\end{frame}

\begin{frame}
\frametitle{Sequences and Convergence} 


Sequence of functions: 
\begin{eqnarray}
\left\{ f_{i} \right\}_{i=1}^{\infty}  & = & \left\{f_{1}, f_{2}, f_{3}, \hdots, f_{n}, \hdots, \right\}\nonumber 
\end{eqnarray}

\begin{defn}
Suppose $f_{i}: X \rightarrow \Re$ for all $i$.  Then $\left\{ f_{i} \right\}_{i=1}^{\infty} $ converges \alert{pointwise} to $f$ if, for all $x \in X$ and $\epsilon> 0$, there is an $N$ such that for all $n\geq N$,  
\begin{eqnarray}
|f_{n} (x)  - f(x)|<\epsilon \nonumber 
\end{eqnarray}
\end{defn}

This is as strong of a statement as we're likely to make in statistics \\



\end{frame}





\begin{frame}
\frametitle{Convergence Definitions}

Define $\widehat{\theta}_{n}$ to be estimator for $\theta$ based on $n$ observations.  \pause  \\
\invisible<1>{\alert{Sequence} of estimators: increasing sample size} \pause \\
\begin{eqnarray}
\invisible<1-2>{\left\{\widehat{\theta}_{i}\right\}_{i=1}^{n} &= & \left\{\widehat{\theta}_{1}, \widehat{\theta}_{2},    \widehat{\theta}_{3}, \hdots, \widehat{\theta}_{n} \right\} \nonumber} \pause  
 \end{eqnarray}

\invisible<1-3>{Question: What can we say about $\left\{\widehat{\theta}_{i}\right\}_{i=1}^{n}$ as $n\rightarrow \infty$? } \pause 
\begin{itemize}
\invisible<1-4>{\item[-] What is the probability $\widehat{\theta}_{n}$ differs from $\theta$?  } \pause 
\invisible<1-5>{\item[-] What is the probability $\left\{\widehat{\theta}_{i}\right\}_{i=1}^{n} $ converges to $\theta$? } \pause 
\invisible<1-6>{\item[-] What is sampling distribution of $\widehat{\theta}_{n}$ as $n \rightarrow \infty$ ? } 
\end{itemize}


\end{frame}


\begin{frame}
\frametitle{Convergence in Probability}

\pause 
\invisible<1>{\begin{defn} 
We will say the sequence $\widehat{\theta}_{n}$ converges in probability to $\theta$ (perhaps a non-degenerate RV)  if, 
\begin{eqnarray}
\lim_{n\rightarrow\infty} Prob(|\widehat{\theta}_{n} - \theta | > \epsilon ) = 0 \nonumber 
\end{eqnarray}
For any $\epsilon>0$
\end{defn}} \pause 

\begin{itemize}
\invisible<1-2>{\item[-] $\epsilon$ is a tolerance parameter: how much error around $\theta$?} \pause 
\invisible<1-3>{\item[-] In the limit, convergence in probability implies sampling distribution collapses on a spike at $\theta$} \pause 
\invisible<1-4>{\item[-] $\left\{\widehat{\theta}_{i}\right\}$ need not actually converge to $\theta$, only P($|\theta_{n}$ - $\theta| > \epsilon) = 0 $ } 
\end{itemize}

\end{frame}


\begin{frame}
\frametitle{Example (Cassella and Burger)} 
\pause 
\invisible<1>{Suppose $S \sim $ Uniform(0,1).  Define $X(s) = s$. } \pause \\
\invisible<1-2>{Suppose $X_{n}$ is define as follows: } \pause 
\begin{eqnarray}
\invisible<1-3>{X_{1}(s) = s + I(s \in [0,1]) & ,&  X_{2} (s)  = s +  I(s \in [0,1/2]) \nonumber \\} \pause
\invisible<1-4>{ X_{3}(s)  = s + I(s \in [1/2, 1])& ,&  X_{4}(s) = s + I(s \in [0,1/3]) \nonumber \\} \pause
\invisible<1-5>{ X_{5} (s)  =  s + I(s \in [1/3,2/3]) & ,&  X_{6}(s) = s + I(s \in [2/3, 1]) \nonumber } \pause
\end{eqnarray}
\invisible<1-6>{Does $X_{n}(s)$ pointwise converge to $X(s)$? \\} \pause
\invisible<1-7>{Does $X_{n}(s) $ converge in probability to $X(s)$?} \pause

\begin{eqnarray}
\invisible<1-8>{P(|X_{n} - X | > \epsilon)  & = &P ( s \in [l_{n}, u_{n} ] ) \nonumber } \pause
\end{eqnarray} 

\invisible<1-9>{Length of $[l_{n}, u_{n}] \rightarrow 0  \Rightarrow P ( s \in [L_{n}, U_{n} ] )   = 0 $} 


\end{frame}





\begin{frame}
\frametitle{Almost Sure Convergence} 

\pause 
\begin{defn} 
\invisible<1>{We will say the sequence $\widehat{\theta}_{n}$ converges almost surely to $\theta$ if, 
\begin{eqnarray}
Prob(\lim_{n \rightarrow \infty} |\widehat{\theta}_{n} - \theta| > \epsilon ) = 0 \nonumber } \pause 
\end{eqnarray}

\end{defn}


\begin{itemize}
\invisible<1-2>{\item[-] Stronger: says that sequence converges to $\theta$ (almost everywhere) ) } \pause 
\invisible<1-3>{\item[-] Think about definition of random variable: $\widehat{\theta}_{n}$ is a function from sample space to real line.} \pause 
\invisible<1-4>{\item[-]  Almost sure says that, for all outcomes ($s$)  in  sample space ($S$) $s \in S$, } \pause 
\begin{eqnarray} 
\invisible<1-5>{\widehat{\theta}_{n}(s) & \rightarrow& \theta(s) \nonumber } \pause 
\end{eqnarray}
\invisible<1-6>{\item[] Except for a subset $\mathcal{N} \subset S $ such that $P(\mathcal{N}) = 0 $.  } 
\end{itemize}

\end{frame}

\begin{frame}
\frametitle{Example (Cassella and Burger)} 

Suppose $S \sim $ Uniform(0,1).\\
Suppose $X_{n}$ is define as follows: 
\begin{eqnarray}
X_{1}(s) = s + I(s \in [0,1]) & ,&  X_{2} (s)  = s +  I(s \in [0,1/2]) \nonumber \\
 X_{3}(s)  = s + I(s \in [1/2, 1])& ,&  X_{4}(s) = s + I(s \in [0,1/3]) \nonumber \\
 X_{5} (s)  =  s + I(s \in [1/3,2/3]) & ,&  X_{6}(s) = s + I(s \in [2/3, 1]) \nonumber 
\end{eqnarray}
Does $X_{n}(s) $ converge almost surely to $X(s) = s$?\pause 

\invisible<1>{\alert{No!}: the sequence doesn't converge for each $s$} \pause \\
\invisible<1-2>{For each value of $s$ the sequence varies between $s$ and $s + 1$ infinitely often } 
\end{frame}







\begin{frame}
\frametitle{Convergence in Distribution}
We've talked about $\widehat{\theta}_{n}$'s sampling distribution converging to a normal distribution.\pause   \\
\invisible<1>{This is \alert{convergence in distribution} }\pause 

\begin{defn}
\invisible<1-2>{$\widehat{\theta}_{n}$, with cdf $F_{n}(x)$,  converges in distribution to random variable $Y$ with cdf $F(x)$ if 
\begin{eqnarray}
\lim_{n\rightarrow \infty} |F_{n} (x) - F(x) | = 0 \nonumber 
\end{eqnarray} 
For all $x \in \Re$ where $F(x)$ is continuous.  } \pause 
\end{defn}

\begin{itemize}
\invisible<1-3>{\item[-] Weakest form of convergence almost sure $\rightarrow$ probability $\rightarrow$ distribution } \pause 
\invisible<1-4>{\item[-] Says that cdfs are equal, says nothing about convergence of underlying RV } \pause 
\invisible<1-5>{\item[-] Useful for justifying use of some sampling distributions } 
\end{itemize}

\end{frame}


\begin{frame}
\frametitle{Convergence in Distribution $\not \Rightarrow$ Convergence in Probability}

Define $X \sim N(0,1)$ and each $X_{n} = - X$.  Then:

$X_{n} \sim N(0,1)$ for all $n$ so $X_{n}$ trivially converges to $X$. But, 
\begin{eqnarray}
P(|X_{n} - X| > \epsilon ) & = & P(|X + X| > \epsilon) \nonumber \\
							& = & P( |2X| > \epsilon) \nonumber \\
						    & =  & P(|X| > \epsilon/2) \not \leadsto 0 \nonumber 
\end{eqnarray}						    





\end{frame}


\begin{frame}
\frametitle{Central Limit Theorem}

\begin{prop}
Let $X_{1}$, $X_{2}, \hdots$ be a sequence of independent random variables with mean $\mu$ and variance $\sigma^2$.  Let $X_{i}$ have a cdf $P(X_{i} \leq x) = F(x)$ and moment generating function $M(t) = E[e^{tX_{i}}]$ .  Let $S_{n} = \sum_{i=1}^{n} X_{i}$.  Then 

\begin{eqnarray}
\lim_{n\rightarrow \infty} P\left( \frac{S_{n} - \mu n }{\sigma\sqrt{n}} \leq x \right) &  = &  \frac{1}{\sqrt{2\pi} } \int_{-\infty}^{x} \exp\left( -\frac{z^{2} }{2} \right) dz \nonumber 
\end{eqnarray}


\end{prop}

\pause 
\invisible<1>{Proof plan:}
\begin{itemize}
\invisible<1>{\item[1)] Rely on Fact that convergence of MGFs$\leadsto$ convergence in CDFs}
\invisible<1>{\item[2)] Show that MGFs, in limit, converge on normal MGF}
\end{itemize}


\end{frame}


\begin{frame}

\begin{prop}
Let $F_{n}$ be a sequence of cumulative distribution functions with the corresponding moment generating functions $M_{n}$.  $F$ be a cdf with the moment generating functions $M$.  If $\lim_{n\rightarrow \infty} M_{n}(t) \rightarrow M(t)$ for all $t$ in some interval, then $F_{n}(x) \leadsto F(x)$ for all $x$ (when $F$ is continuous).  
\end{prop}

\pause 
\invisible<1>{\begin{prop}
Suppose $\lim_{n\rightarrow \infty} a_{n} \rightarrow a$, then 
\begin{eqnarray}
\lim_{n\rightarrow \infty} \left( 1 + \frac{a_{n}}{n}\right)^{n}  & = & e^{a} \nonumber 
\end{eqnarray}

\end{prop}
}

\pause 

\invisible<1-2>{\begin{prop}
Suppose $M(t)$ is a moment generating function some random variable $X$.  Then $M(0) = 1$.  
\end{prop}}



\end{frame}




\begin{frame}
\frametitle{Proof of Central Limit Theorem (Courtsey of Swarthmore Notes)}

\large{Proof.}
Suppose $X_{1}, \hdots, X_{n}$ are iid variables with $E[X] = 0 $, variance $\sigma^{2}_{x}$, Moment Generating Function (MGF) $M_{x}(t)$. \pause  \\

\invisible<1>{Let $S_{n} = \sum_{i=1}^{n} X_{i}$ and $Z_{n} = \frac{S_{n}}{\sigma_{x} \sqrt{n}}$. } \pause \\
\invisible<1-2>{$M_{S_{n}}  = (M_{x}(t))^{n} $ and $M_{Z_{n}} (t) = \left(M_{x} \left(\frac{t}{\sigma_{x} \sqrt{n}}    \right)    \right)^{n}$} \pause \\
\invisible<1-3>{Using Taylor's Theorem we can write
\begin{eqnarray}
M_{x} (s) & = & M_{x} (0)  + s M^{'}_{x}(0) + \frac{1}{2} s^2 M_{x}^{''}(0) + e_{s} \nonumber 
\end{eqnarray}
$e_{s}/s^2 \rightarrow 0 $ as $s\rightarrow 0$.  } 

\end{frame}


\begin{frame}
\begin{eqnarray}
M_{x} (s) & = & M_{x} (0)  + s M^{'}_{x}(0) + \frac{1}{2} s^2 M_{x}^{''}(0) + e_{s} \nonumber 
\end{eqnarray}

Filling  in the values we have 
\begin{eqnarray}
M_{x} (s) & = & 1  + 0 + \frac{\sigma_{x}^{2}}{2} s^2  + \underbrace{e_{s}}_{\text{Goes to zero}} \nonumber 
\end{eqnarray}

Set $s = \frac{t}{\sigma_{x} \sqrt{n}} $ $\lim_{n\rightarrow \infty} s \rightarrow 0$.  Then 

\begin{eqnarray}
M_{Z_{n}}(t) & = & \left(1 + \frac{\sigma_{x}^{2}}{2}\left( \frac{t}{\sigma_{x} \sqrt{n}} \right)^{2} \right)^{n} \nonumber \\
& = & \left( 1 + \frac{t^{2}/2}{n} \right)^{n} \nonumber \\
\lim_{n\rightarrow \infty} M_{Z_{n}}(t) & = & e^{\frac{t^2}{2}}\nonumber 
\end{eqnarray}

\end{frame}









\end{document}
