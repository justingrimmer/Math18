\documentclass[12pt]{article}
\usepackage{amsmath,amssymb}
\newtheorem{theorem}{Theorem}
\newtheorem{proposition}[theorem]{Proposition}
\newtheorem{corollary}[theorem]{Corollary}
\newtheorem{lemma}[theorem]{Lemma}
\usepackage[margin=2.5cm]{geometry}

\renewcommand{\labelenumi}{(\alph{enumi})}
\begin{document}

\centerline{\bf Math Camp - Problem Set 8 Solutions}

\bigskip
The following laws of set algebra will be useful:\\
\bigskip

Commutative Law:\\
$A\cup B = B \cup A$\\
$A\cap B = B \cap A$\\

Associative Law:\\
$(A\cup B) \cup C = A\cup (B \cup C)$\\
$(A\cap B) \cap C = A\cap (B \cap C)$\\


Distributive Law:\\
$A\cup (B\cap C) = (A\cup B) \cap (A \cup C)$\\
$A\cap (B\cup C) = (A\cap B) \cup (A \cap C)$\\

\bigskip
De Morgan's Law:\\
$(A \cap B)^c = A^c\cup B^c$\\


\bigskip

\noindent \textbf{Question 1}: Prove that if $A \subset B$ and $C \subset D$, then $A \cup C \subset B \cup D$ and $A \cap C \subset B \cap D$. (\textit{Hint:} This can be proven either directly or by contradiction.)

\bigskip
For any two events, $A$ and $B$, if $A$ is a subset of $B$ (denoted as $A\subset B$), then every element contained in event $A$ is also contained within event $B$.\\
1) If $A \subset B$, then $A \cup B=B$ (definition of a subset)\\
2) If $C \subset D$, then $C \cup D=D$ (definition of a subset)\\
3) From 1 and 2, it follows that:\\

$(A \cup B) \cup (C \cup D) = B \cup D$ \\


 $(A \cup C) \cup (B \cup D) = B \cup D$\\
 
 \bigskip
4) if $ (A \cup C)\cup (B \cup D) = B \cup D$, then $ (A \cup C)\cup (B \cup D) \subset B \cup D$. $A \cup C$ and $B \cup D$ are events, so we can apply the same principle from (1) and (2).\\

\bigskip

\noindent \textbf{Question 2:} $A$, $B$, and $D$ are non-mutually exclusive events contained within a sample space $S$. Find the simplest form for the following set expressions:
\begin{enumerate}
\item $(A \cap B) \cup (A \cap B^c)$
\item $(A \cap B) \cap (A \cap B^c)$
\item $(A \cap B) \cap (A^c \cup B)$
\item $(A \cap B) \cap (B \cap D)$
\item $(A^c \cup B^c \cup D^c)^c$
\item $(A \cup B) \cap (B \cup D)$
\end{enumerate}
\bigskip

a)\\
$(A \cap B) \cup (A \cap B^c) = [(A \cap B) \cup A] \cap [(A \cap B) \cup B^c] (\mbox{distributive law})$\\
$A \cap [(A \cup B^c) \cap (B \cup B^c)](\mbox{distributive law})$\\
$A \cap [(A \cup B^c) \cap S](\mbox{distributive law})$\\
$A \cap (A \cup B^c) $\\
=$A$\\

b)\\
$(A \cap B) \cap (A \cap B^c) = (A \cap B \cap A) \cap (A \cap B \cap B^c) (\mbox{distributive law})$\\
 $= (A \cap B) \cap \emptyset$\\
 $= \emptyset$\\
 
 c)\\
$(A \cap B) \cap (A^c \cup B) = (A \cap B \cap A^c) \cup (A \cap B \cap B) (\mbox{distributive law})$\\
 $= \emptyset \cup (A \cap B)$\\
 $= A \cap B$\\
 
  d)\\

$(A \cap B) \cap (B \cap D) = (A \cap B \cap B) \cap (A \cap B \cap D)$ by distributive law\\
=$(A \cap B) \cap (A \cap B \cap D) $\\
=$A \cap B  \cap D $\\



  e)\\
$(A^c \cup B^c \cup D^c)^c$ \\
We know that  $(A\cup B)^c = A^c \cap B^c$ (by De Morgan's law), so it must also be true that $(A^c \cup B^c)^c = A \cap B$. Applying the compliment to both sides of the equation, we get $A^c \cup B^c = (A \cap B)^c$.\\

($A^c \cup B^c \cup D^c)^c = [(A \cap B)^c \cup D^c]^c$\\
$= [(A \cap B \cap D)^c]^c$\\
$= A \cap B \cap D$\\

  f)\\
$(A \cup B) \cap(B \cup D)  = [(A \cup B) \cap B] \cup [(A \cup B) \cap D]$ (distributive law) \\
=$B \cup [(A \cap D) \cup (B\cap D)]$(distributive law) \\
=$B \cup (B \cap D) \cup (A\cup D)$ \\
=$B \cup (A \cap D)$ \\

\medskip

\noindent \textbf{Question 3}: Are the following statements true or false? Explain. (\textit{Hint:} The inclusion-exclusion principle might be useful.)
\begin{enumerate}
\item If I flip a coin $n$ times, the probability of getting fewer than $m$ heads is equal to the sum of the probability of getting $k$ heads for all integers $0 < k < m$.
\item Suppose I roll six fair, ordinary dice. Let $E_1$ be the event in which I rolled exactly one 1, $E_2$ be the event in which I rolled exactly one 2, $E_3$ be the event in which I rolled exactly one 3, and so on through $E_6$. Then: $$P\left( \bigcup_{i=1}^{6}E_{i} \right) = P(E_1)+P(E_2)+P(E_3)+P(E_4)+P(E_5)+P(E_6)$$
\end{enumerate}

\bigskip

a) Let the random variable $H$ be the number of heads obtained in $n$ coin flips. First, figure out the probability that we get fewer than $m$ heads out of $n$ flips, which is $P(H<m)$. This is the probability that we get anywhere from 0 to $m-1$ heads (remember the probability of the union of disjoint events is the sum of the probabilities of each event), which can be expressed as:\\

$P(H <m) = P(H = 0 \cup H=1 \cup \ldots \cup H = m-1)$\\
=$ P(H = 0) + P(H=1) + \ldots \cup H=1 \cup \ldots +  P( m-1)$\\
=$ \sum_{k=0}^{m-1} P(H_k) $\\
=$ \sum_{0 \leq k < m} P(H_k) $\\

The question asks if $P(H<m)$ is equal to the sum of the probability of getting anywhere from 1 to $m-1$ heads (all outcomes greater than 0 or less than $m$).

$ \sum_{0 < k<m} P(H_k)= \sum_{k=1}^{m-1} P(H_k)$\\

The statement is false. The probability of getting fewer than $m$ heads in $n$ flips is the sum of the probability of getting all outcomes of heads from 0 to $m-1$. The sum of the probability of getting $k$ heads for all integers $0<k<m$ is the sum of the probability of getting all outcomes of heads from 1 to $m-1$. The second quantity leaves out the probability of getting zero heads, and thus, the two statements are not equivalent.  


\bigskip

b) By the inclusion-exclusion principle, the above statement is true if the events $E_1$ to $E_6$ are disjoint (do not share any elements n the sample space). However, consider when the outcome of the rolls is 1,2,3,4,5,6. This outcome would be an element of $E_1$ through $E_6$ meaning these events are not disjoint, so the statement is false. 
\bigskip

\noindent \textbf{Question 4:} Events $A$ and $B$ are contained within a sample space $S$. Given that $P(A)=0.5$, $P(B)=0.3$ and $P(A \cap B) = 0.1$, find:
\begin{enumerate}
\item $P(A \cup B)$
\item $P(A \cap B^c)$
\item $P[(A \cap B^c) \cup (B \cap A^c)]$
\end{enumerate}
\medskip
(\textit{Hint:} The inclusion-exclusion principle might be useful here, as well.)

\bigskip


a) $P(A \cup B) = P(A) + P(B) - P(A \cap B) = 0.5+0.3 - 0.1 = 0.7$\\
\bigskip


b) $ P(A) = P(A \cap B) + P(A \cap B^c)$ \\
 $ P(A \cap B^c) = P(A) - P(A \cap B)$ \\
 =$ 0.5 - 0.1 = 0.4$ \\

\bigskip
c) $ P[(A \cap B^c) \cup (B \cap A^c)] = P(A \cap B^c) + P(B \cap A^c) - P(A \cap B^c \cap B \cap A^c)$\\
$= P(A \cap B^c) + P(B \cap A^c) - P(\emptyset)$ \\
$= P(A) - P(A\cap B) + P(B) - P(A \cap B)$ \\
$=0.5 - 0.1 + 0.3 - 0.1 = 0.6$ \\


\noindent \textbf{Question 5:} A political campaign in New Haven, CT.\ decides to conduct an ``experiment" to determine the effectiveness of knocking on a door in turning a resident of that house out to vote. The campaign foolishly denies an offer from a team of political scientists to help them design a protocol for this experiment, and instead directs their two teams of volunteers to each select a random group of the 120 total houses in the district and to go knock on as many of those random doors as they can in the week before the election. The campaign manager directs the teams to count the number of doors on which they knock and to record the names of the residents who live in each house, but neglects to ensure that the two teams select a mutually exclusive set of houses, or to set bounds on how many houses each team chooses. 

On election day, the Team 1 members return, and proudly report to the campaign manager that they knocked on 70\% of the doors in the electoral district. The Team 2 members return shortly after, and report that they knocked on 40\% of the doors in the electoral district. In looking over the names the teams recorded, the campaign manager quickly determines that not only was every house in the district contacted, but some houses were contacted by both teams. (This will make drawing inferences about the effectiveness of door knocking\ldots difficult.)

Use what we have learned about probability to determine how many houses had their doors knocked on by both teams.

\bigskip
Answer:

Let $A$ be the event that Team 1 knocked on a given door and $B$ be the event that Team 2 knocked on a given door. We know that\\

$P(A\cup B) = 1$ since every house had their door knocked on at least once. Further, we know that:

$P(A \cup B) = Pr(A) + Pr(B) - Pr(A \cap B)$\\

Plugging in, we get\\

$1 = .7 + .4 - Pr(A \cap B)$\\
$-.1 = - Pr(A \cap B)$\\
$.1 = Pr(A \cap B)$\\

Since there are 120 doors in the district, this means that $.1*120=12$ doors were knocked on by both teams.

 
%
%
%\noindent \textbf{Question 5:} There are currently 193 members of the United Nations General Assembly. The United Nations Security Council is composed of 15 countries in the General Assembly. Five of those members (China, France, the USA, Russia, and the UK) hold permanent places on the Security Council; the remaining 10 spaces on the Security Council are filled by a selection from the remaining 188 members of the General Assembly. Pretending for this problem that there are no restrictions on which groups of 10 countries can fill those spaces, how many possible Security Councils are there? Use \texttt{R} to determine an actual numerical answer; do not simply give a formula. That said, answers in scientific notation are perfectly fine.
%
%\bigskip
%
%\noindent \textbf{Question 6 (Challenging):} We would like to know what the probability is that \textit{exactly} two people in a class of 45 people share the same birthday. Rather than find a precise solution, we wish to use a Monte Carlo simulation to determine this probability. We will do this by simulating the birthdays for a random class of 45 people. Then, we will determine whether two people in the class share the same birthday. Finally, we will repeat this a large number of times and calculate the fraction of simulated classes where exactly two people shared the same birthday. For this problem, we will assume that there is an equal probability of being born on any day of the year and that we are dealing with regular 365 day years.
%
%\begin{enumerate}
%\item First, we will simulate the birthdays of a class of 45 people using the \texttt{sample} function in \texttt{R}. Enter \texttt{?sample} to see the documentation for this function. We will need to give 3 inputs to the sample function: \texttt{x}, the data to be sampled; \texttt{size}, the size of the sample to draw; and the option \texttt{replace=TRUE} so that we sample with replacement. Use the \texttt{sample} function to simulate the birthdays of 45 people.  (\textit{Hint:} You might find it useful to list birthdays as integers from 1 to 365.)
%
%\item Your \texttt{sample} function should return a vector of birthdays for your simulated class. Now, check how many duplicate birthdays you have in your simulated class. Use the \texttt{unique} function to do this. Set the variable \texttt{duplicatebirthdays} to be the number of duplicate birthdays in your sample.
%
%\item Next, create a \texttt{for} loop to repeat this process a large number of times; say 30,000. The \texttt{for} loop will repeat the steps in parts (a) and (b) many times and also save the output (number of duplicated birthdays). We will save the number of duplicate birthdays in the $i$-th sample class as the $i$-th element of the vector \texttt{duplicatebirthdays}. Use the \texttt{R} code below as a template:
%
%\pagebreak
%
%\begin{verbatim}
%##n is the number of times you want to simulate sample classes
%n <- 30000
%
%##We must create the variable "duplicatebirthdays"
%##before we can save anything into it from a for loop
%duplicatebirthdays <- rep(NA, n)
%
%for(i in 1:n){
%	##Code from parts (a) and (b)
%	duplicatebirthdays[i] <- #Number of duplicate b-days in a sample class
%	}
%\end{verbatim}
%\item What is the probability that exactly two people in a class of 45 people share the same birthday?
%\item What is the probability that more than a single pair of people has the same birthday in a class of 45 people?
%
%\end{enumerate}

\end{document}